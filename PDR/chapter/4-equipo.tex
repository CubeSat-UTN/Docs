\chapter{Datos del Equipo y Unidad Académica}

\definecolor{celeste}{RGB}{121,170,223}
\noindent
\resizebox{\textwidth}{!}{
\begin{tabular}{
|>{\centering\arraybackslash}m{5.5cm}|
 >{\centering\arraybackslash}m{3cm}|
 >{\centering\arraybackslash}m{4.5cm}|
 >{\centering\arraybackslash}m{2.4cm}|
 >{\centering\arraybackslash}m{5cm}|
 }
\hline
\multicolumn{5}{|c|}{\cellcolor{celeste}\textbf{CONCURSO CUBESAT UTN 2025}} \\
\hline
\multicolumn{5}{|c|}{\cellcolor{cyan!10}\textbf{REVISIÓN DE DISEÑO PRELIMINAR}} \\
\hline
\multicolumn{1}{|l|}{\textbf{FACULTAD REGIONAL:}} & \multicolumn{4}{l|}{Córdoba} \\
\hline
\multicolumn{1}{|l|}{\textbf{NOMBRE DEL EQUIPO:}} & \multicolumn{4}{l|}{} \\
\hline
\multicolumn{5}{|l|}{\cellcolor{cyan!20}\textbf{INTEGRANTES DEL EQUIPO:}} \\
\hline
\textbf{ALUMNO} & \textbf{AÑO} & \textbf{CARRERA} & \textbf{LEGAJO} & \textbf{CORREO} \\
\hline
Adragna, Jimena Sofía & & Ingeniería en Sistemas de Información & 94269 & jsadragna@gmail.com \\
\hline
Cortesini Perez, Luciano Tomas & 3ro & Ingeniería Electrónica & 402719 & cortesiniluciano@gmail.com\\
\hline
Gil, Ignacio & 3ro & Ingeniería Electrónica & 401891 & cortesiniluciano@gmail.com\\
\hline
& & & & \\
\hline
& & & & \\
\hline
\multicolumn{5}{|l|}{\cellcolor{cyan!20}\textbf{MENTOR PRINCIPAL}} \\
\hline
\textbf{NOMBRE} & \textbf{CARGO} & \textbf{MATERIA} & \textbf{CARRERA} & \textbf{CORREO} \\
\hline
& & & & \\
\hline
\multicolumn{5}{|l|}{\cellcolor{cyan!20}\textbf{MENTOR SUPLENTE (Opcional)}} \\
\hline
\textbf{NOMBRE} & \textbf{CARGO} & \textbf{MATERIA} & \textbf{CARRERA} & \textbf{CORREO} \\
\hline
& & & & \\
\hline
\multicolumn{5}{|l|}{\cellcolor{cyan!20}\textbf{AUTORIDAD RESPONSABLE DEL PROYECTO EN REGIONAL}} \\
\hline
\textbf{NOMBRE} & \textbf{CARGO} & \multicolumn{3}{c|}{\textbf{CORREO}} \\
\hline
& & \multicolumn{3}{c|}{asdajj} \\
\hline
\end{tabular}
}

\section{Presentación del Equipo}

\subsection{Integrantes}

\presentacion
  {Adragna, Jimena Sofía}
  {Programación en Python y JavaScript\\Dominio en el análisis, diseño y modelado de sistemas de información}
  {Pentester en Grupo de Investigación en Seguridad de Sistemas de Información y Ciberseguridad (GISSIC), UTN-FRC}
  {image/portrait/cortesini.jpg}

\presentacion
  {Cortesini Pérez, Luciano Tomás}
  {Impresión 3D con materiales técnicos\\Diseño 3D paramétrico\\Programación en C++ y Python\\Gestión de repositorios Git}
  {Finalista Certamen CanSat Argentina 2022\\Desarrollador de software en el proyecto HexSar del Centro de Investigación de Informática para la Ingeniería}
  {image/portrait/cortesini.jpg}

\presentacion
  {Gil, Ignacio}
  {Diseño PCB\\Programación en C, C++ y Python\\Gitflow\\Diseño 3D}
  {Finalista Certamen CanSat 2022\\Técnico Electrónico egresado del ITS Villada\\Omixom SRL}
  {image/portrait/cortesini.jpg}

\presentacion
  {Koroch, Matías Adolfo}
  {Programación en Python, JavaScript y Haskell\\Inteligencia Artificial Generativa y análisis de datos\\Evaluación de LLM y desarrollo e implementación de pipelines automatizados de benchmarking\\Dominio de bibliotecas como LangChain, TensorFlow y Scikit-learn}
  {Becario de investigación en inteligencia artificial en UTN-FRC\\Autor de una publicación presentada en CONAIISI 2024\\Diseño e implementación de un sistema automatizado para evaluar LLM en resolución algorítmica}
  {image/portrait/cortesini.jpg}

\presentacion
  {Montesinos, Dana Carolina}
  {Programación en Python y JavaScript\\Física computacional\\Programación en Fortran\\Estudiante de Licenciatura en Física}
  {Participación en stands de divulgación científica orientados a la comunidad}
  {image/portrait/cortesini.jpg}

\presentacion
  {Palombo, Franco}
  {Diseño y fabricación de circuitos electrónicos impresos\\Programación en C y C++ para sistemas embebidos y aplicaciones de escritorio\\Diseño 3D paramétrico\\Impresión 3D}
  {Finalista Certamen CanSat 2022\\Técnico Electrónico egresado del ITS Villada\\Desarrollador de software en el proyecto HexSar del Centro de Investigación de Informática para la Ingeniería}
  {image/portrait/cortesini.jpg}

\presentacion
  {Prieto, Angelo}
  {Programación en C, C++ y Python\\Programación de sistemas embebidos\\Fabricación de circuitos electrónicos impresos\\Fotografía y edición de video para documentación y divulgación de proyectos}
  {Diseño e implementación de proyectos de hardware y sistemas embebidos}
  {image/portrait/cortesini.jpg}

\subsection{Mentores}
\presentacion
  {Paz, Claudio}
  {Programación\\Sistemas embebidos\\Robótica\\Visión Artificial}
  {+10 años en Dirección de proyectos de I+D+i}
  {image/portrait/cortesini.jpg}

